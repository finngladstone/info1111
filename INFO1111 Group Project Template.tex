\documentclass[a4paper, 11pt]{report}
\usepackage{blindtext}
\usepackage[T1]{fontenc}
\usepackage[utf8]{inputenc}
\usepackage{titlesec}
\usepackage{fancyhdr}
\usepackage{geometry}

\usepackage[english]{babel}
\usepackage{apacite}

\geometry{ margin=30mm }
\counterwithin{subsection}{section}
\renewcommand\thesection{\arabic{section}.}
\renewcommand\thesubsection{\thesection\arabic{subsection}.}
\usepackage{tocloft}
\renewcommand{\cftchapleader}{\cftdotfill{\cftdotsep}}
\renewcommand{\cftsecleader}{\cftdotfill{\cftdotsep}}
\setlength{\cftsecindent}{2.2em}
\setlength{\cftsubsecindent}{4.2em}
\setlength{\cftsecnumwidth}{2em}
\setlength{\cftsubsecnumwidth}{2.5em}


\begin{document}
\titleformat{\section}
{\normalfont\fontsize{15}{0}\bfseries}{\thesection}{1em}{}
\titlespacing{\section}{0cm}{0.5cm}{0.15cm}
\titleformat{\subsection}
{\normalfont\fontsize{13}{0}\bfseries}{\thesubsection}{0.5em}{}
\titlespacing{\section}{0cm}{0.5cm}{0.15cm}

%=======================================================================================

\begin{titlepage}
\center 
\textbf{\huge INFO1111: Computing 1A Professionalism}\\[0.75cm]
\textbf{\huge 2022 Semester 1}\\[2cm]
\textbf{\huge Practice: Team Project Report}\\[3cm]

\textbf{\huge Submission number: 01}\\[0.75cm]
\textbf{\huge Team Members:}\\[0.75cm]
\textbf{\large
    \begin{tabular}{|p{0.5\textwidth}|p{0.3\textwidth}|p{0.2\textwidth}|}
        \hline
        Name & Student ID & Levels being attempted in this submission\\
        \hline
        Finn Gladstone & 510448570 & 1,2 \\
        Anh Hoo & 510442958 & 1,2 \\
        Adam Jankelowitz & 510441180 & 1,2 \\
        ?? & ?? & ?? \\
        \hline
    \end{tabular}
}\\[0.75cm]
\end{titlepage}

%=======================================================================================

\tableofcontents

%=======================================================================================

\newpage
\section*{General Instructions}

You should use this \LaTeX\ template to generate your team project report. Keep in mind the following key points:
\begin{itemize}
    \item When we assess your report, you are not given a mark. Instead we will indicate (separately, for each team member) whether each level is ''achieved''.
    \item In order to pass the unit, you must achieve at least level 1. 
    \item In order to achieve level 2, you must first have achieved level 1, and so on for each level up to level 4. This means that we will not assess a higher level until a lower level has been achieved (though we will review one level higher and give you feedback to help you in refining your work).
    \item Some parts of the report are completed as a team and other parts require each student to complete a different section. This means that for each submission, some members of the team may have completed their work for a given section, but other members may not. It also is therefore possible that some members of the team may achieve a specified level and other members of the team may not yet have achieved that level.
    \item Even if some members are completing their material for a given level, and others are not, your team members will still need to work together to edit and compile the report.  The only exception to this is where a member of the team has already achieved the level they are targeting in a previous submission and has decided to not attempt higher levels, and so is not contributing any further (this should be obvious because no level is indicated for that student on the cover page).
    \item When completing each section you should remove the explanation text and replace it with your material.
\end{itemize}

For each submission you will add new details to this report, and/or update previous sections (where previous work was not good enough to have achieved the relevant level). In particular:

\begin{itemize}
    \item \textbf{General:} For each submission, each student can attempt up to 2 levels. You must also successfully achieve each lower level before you can be assessed at a higher level. For example, in the first submission you might attempt only level 1, but not be successful in achieving that level. You then reattempt level 1 and add in level 2 in the second submission and are successful in achieving level 1 but not level 2. For the third and final submission you could then attempt level 2, or levels 2 and 3 - or even just choose to not submit anything further and remain at level 1).
    \item \textbf{Submission 1:} You should complete at least the material for level 1 (since achieving level 1 is required to pass the unit). Each member of the team can also optionally choose to complete the material for level 2.\\
    \textit{Note 1: If you do not complete the level 2 information then you obviously cannot achieve level 2 at this stage. This does not stop you from attempting level 2 in Deliverable 2 or 3, but it will make it more difficult to achieve the higher levels later in the semester.}
    \textit{Note 2: To be able to achieve Level 1 in submission one your team has to achieve level 1 in the group component (Section 1.1) and you have to achieve Level 1 in the individual component (i.e. your assigned section 1.2, 1.3, 1.4 or 1.5)}
    \item \textbf{Submission 2:} Each member of your team will complete additional sections, but because you are submitting a single document, you need to work together to compile your results together and generate the final submission.\\
    If you did not achieve level 1 in your first submission, then you should revise the material for level 1 based on the feedback, and optionally you can also complete level 2.\\
    If you achieved level 1 in your first submission, then each team member can optionally complete the material for levels 2 and 3.
    \textit{Note: If you do not achieve level 1 with this submission then the highest level you will be able to achieve in the final submission will be level 2. If you achieve level 1, but not level 2, with this submission then the highest level you will be able to achieve with the final submission is level 3.}
    \item \textbf{Submission 3:} Again, you can correct sections where you did not achieve the specified level in the previous submission, and you complete additional sections.\\
    If you still have not achieved level 1, then you should revise the material for level 1 based on the feedback, and again optionally you can also complete level 2.\\
    For those at level 1, you can choose to complete the material for levels 2 and 3.\\
    For those at level 2, you can choose to complete the material for levels 3 and 4.\\
    For those at level 3, you can choose to complete the material for level 4.
\end{itemize}

Whilst the team project is just that -- a team project -- it has been designed to also allow different members of the team to achieve different outcomes. We do expect you to work together as a team. If you do come across problems working together then the first step should be to discuss this with your tutor. Note: If you are having problems you should approach your tutor as soon as you can to make them aware of the difficulties you are having with your team.

Finally, you should also ensure that any resources you use are suitably referenced, and references are included into the reference list at the end of this document. You should use APA 6th reference style \cite{apa6}.

%=======================================================================================

\newpage
\section{Level 1: Basic Skills}

Level 1 focuses on basic technical skills (related to \LaTeX\ and Git) and the types of skills used in different computing jobs.

\subsection{Developing industry skills}

Five approaches to continual learning:

\begin{description}

\item [Personal projects in unfamiliar territory:] Undertaking a personal interest project in a new language or package is an effective example of continual learning. \\In the project format, the user needs to not only obtain a certain technical proficiency but be able to transfer those skills into a real-world program (Ikechukwu, 2020).
\\This approach would be effective when the user needs to achieve a high technical proficiency in a certain area, moreso in niche spaces (for e.g., derivations of neural networks; RNN, CNN, ANN etc). 
\\The main downside of this approach is the lack of external motivation or a mentor, and the user needs to direct themselves on project timeframe, work efficiency.  

\item [2. Online bootcamps \& certs:] Online coding workshops are aimed at equipping the user with some of the most in-demand skills in the industry in a time-efficient manner (Columbia, 2019). 
\\Bootcamps teach a variety of skills, such as popular front end lanuages (HTML, CSS, JavaScript) all the way to back end development (SQL, APIs). 
\\Bootcamps are advantageous in their fast-pace environment and use of a structured learning path for the user. 
\\However, some fail to teach more conceptual skills, such as Data Structures \& algos (Shipley, 2019), and they generally come with a high upfront cost.  

\item [3. Conversations \& questions in the workplace:] Asking more questions and observing mentors in the workplace, coined as Social Learning, is a more dynamic and casual process through which individuals can work to continuously improve their skillset (Eira, 2019). 
\\This is an effective method for continuous learning in a less strict or confining environment, and one that helps to foster soft skills (communication, team dynamics, collaborative problem solving). 
\\Hence this approach might lack the technical comprehensiveness of a course but has the side benefit of training the user's soft skills in the workplace, which are crucial for success (Erstad, 2017).

\item [4. Going to seminars and talks:] Seminars are a great way of being introduced to new projects, techniques and industry news.
\\They allow the user to gain more knowledge specific to their profession and stay on the bleeding edge of industry progression, as well as network and get introduced to new fields and peers (Allconference, 2021). 
\\Seminars also provide a learning environment that highly differentiates itself from the classroom (Panigrahi). 
\\Therefore seminars are effective in keeping users up-to-date and providing networking opportunities, but can lack a comprehensive technical facet.

\item [5. Reading articles and journals:] Rote reading is another method to continually learn about one's field. 
\\This includes a variety of media (books, scientific journals, news edits) and presents an easily and effectively digested source of information.
\\While reading is effective to gather a deep conceptual or theoretical understanding of a certain topic, follow-up practical exercises would be necessary to achieve the greatest learning outcome. 


\end{description}


\subsection{Skills: Anh Hoo : Computer Science}

This section is completed individually. Each member of the team should independently complete a separate copy of this section.\\
You should begin by allocating to each team member a different major to focus on (i.e. one of: Computer Science; Data Science; Software Development; Cyber Security). \textit{If you have a fifth member, then your tutor will suggest a fifth topic to cover}. You should then undertake research into the typical practical skills that you believe would be most important to someone who graduates with this major and is then working in industry. You should list the 8 skills that you believe are most important and for each one give a short explanation as to why you feel it is important. (Target = $\sim$100 words per skill $\sim$800 words total per student).

\subsection{Skills: Adam Jankelowitz : Data Science}

\paragraph{SQL Queries}
\\The ability to write SQL Queries and build data pipelines is a crucial skill for a data science graduate. The first reason why this skill is important is that the graduate will be flexible, having the ability to build fundamental pipelines, thus allowing one to improve insight into the data derived (Shin, 2022) Another reason that this is a key skill for a graduate is to gain independence. With this skill, one will be able to write queries for a project or model that does not exist. This makes the graduate more valuable for employees as their independence and ability to not rely on others’ previous work will prove to be efficient for the graduate.

\paragraph{Teamwork} Teamwork is another critical practical skill that a graduate who majors in data science should have. As a data scientist, one will be working with business executives to devise strategies, working as part of a team to create a new product, and also working alongside software developers to improve efficiency and create data pipelines and structures. These different tasks require strong teamwork skills in order to achieve positive outcomes for the company. Whilst working as a data scientist, one will face many challenges, so the ability to work as a team and collaborate, as well as listening to one another will guide one to be successful whilst working in the industry of data science. 

\paragraph{Version Control} Version Control, and specifically the ability to use Git is a key skill that a data science graduate should have. Git is a cloud-based repository that can store files and folders, being one of the main version control systems used worldwide (Shin, 2022)
. Git is an essential practical skill that one should know as it allows collaboration, the ability to bounce off each other’s ideas, and also the capability to work on projects in conjunction with others. Furthermore, Git will also allow the entire team to keep track of old versions of their project, the productivity of the individual/team, and also allows one to go back to older versions of the project if required. 

\paragraph{Programming Knowledge} Another key skill that a graduate of a data science major should have is programming knowledge. Python is one of the most popular programming languages that is used by data scientists due to its easy-to-use nature as well as its access to multiple libraries which are suited for data science projects. If a graduate has a good sense of programming knowledge, they will be able to use different platforms with emerging technologies, enabling one to be efficient when developing programs and projects. A data scientist with a background in programming will be extremely self-sufficient, not relying on others and outside resources to complete tasks and work with data sets. For example, with the knowledge of programming, one will be able to query data individually, without requiring to collaborate with a software engineer. 

\paragraph{Data Wrangling} Data wrangling is the process of cleaning and unifying complex data sets to ensure easy access, readability, and analysis. In the workplace, a data scientist will receive large data sets, sometimes being stored in different formats or on different devices. Due to this, one will require to merge the data into one clean set, ensuring accessibility and readability. Furthermore, if a business wishes to analyse incomplete or unclean data, the analysis made will be inefficient and unreliable. Despite sometimes being time-consuming, data wrangling is a crucial skill that a graduate requires as it ensures the data is reliable and readable before being analysed. Therefore, this practical skill is crucial for a data science graduate and will be used constantly when working in the data industry. 

\paragraph {Statistics} Statistics and statistical analysis are crucial skills that one should know in the modern world. The ability to predict, estimate and analyse trends is a key skill that a data science graduate should know, and will use every day in the industry. Through the knowledge of statistics, one will have the power to derive key insights and solve problems in all industries (University of Virginia, 2022). The ultimate goal of a data scientist is to create value and gain more insight out of data. This can only be achieved by understanding the trends in the data extremely well. In the industry, finding structure and making predictions are the most important steps in data science (Weihs & Ickstadt, 2018)
). One branch in statistical analysis  is classification methods, which is the principal for finding and predicting subpopulations from data. This key concept will be used constantly in the industry along with regression and statistical modelling. 

\paragraph {Data Visualisation} Data Visualisation is the ability to transfer data into a visual image such as graphs or charts. This skill makes the data readable and makes it easier to comprehend, giving one the ability identify trends and outliers within data sets. As a data science graduate working in the industry, data visualisation is a key skill that employers will be looking for as they will require an efficient way to analyse data. (ANALYTIKS) Data visualisation has a positive affect on the decision making of a company as the data is represented visually, allowing companies to recognise trends quickly. Without the ability to express data in visual form, one may find it challenging to identify relationships within data sets, as well as trends that may form over time. Therefore, a graduate who has the skill of data visualisation will be extremely successful in the industry as data visualisation ‘represents one of the fundamental tools of modern data science (Unwin, 2020).

\paragraph {Intellectual Curiosity} Intellectual Curiosity is a crucial, non-technical skill that a graduate who majored in data science will require when working in the industry. When working as a data scientist, one will need to be able to ask questions in order to gain further insight into the specific project or data they are analysing. Intellectual curiosity is defined as “one’s desire to acquire more knowledge” (9 Must-have skills you need to become a Data Scientist, updated - KDnuggets, 2022). Data science is an area that is constantly evolving and in order to keep up with new advancements, one must be inquisitive and seek to constantly update their knowledge through research. Furthermore, curiosity will enable one to search through the data to find answers in multiple ways, as well as gain more insights into the trends that have been analysed. 

\subsection{Skills: Finn Gladstone : Software Development}
fs
Your text goes here

\subsection{Skills: add student 4 name here : Cyber Security}

Your text goes here


%=======================================================================================

\newpage
\section{Level 2: Basic Technology}

Level 2 focuses on initial evaluation of the tech stack that is used by a selected company. All companies make use of a range of technologies, and these technologies need to work together. A tech stack is basically just this collection of technologies that collectively enable a company's systems. As an example, one of the most common technology stacks for supporting web servers is LAMP: Linux as the underlying operating system; Apache as a web server; MySQL as the supporting database; and Perl (or more recently PHP or Python) as the programming language.

Each student should choose a different tech stack and explain the role of each of the different technologies in that stack. Note that prior to researching your proposed tech stack and spending time writing about it, it might be a good idea to check with your tutor as to whether your chosen stack is suitable. (Target = $\sim$200-400 words per student).

\subsection{Tech Stack: Adam Jankelowitz}

Your text goes here

\subsection{Tech Stack: add student 2 name here}

Your text goes here

\subsection{Tech Stack: add student 3 name here}

Your text goes here

\subsection{Tech Stack: add student 4 name here}

Your text goes here


%=======================================================================================

\newpage
\section{Level 3: Advanced Skills}

Level 3 focuses on more advanced technical skills (\LaTeX\ and Git) and analysis of linkages and relationships between the items in the company tech stack.

The following is a list of advanced Git and \LaTeX\ skills/features. Each student should select one pair of items from each list and demonstrate actual use of each item (either through activity in Git, or through including items in this report). (Target = $\sim$100 words per student for each feature).
\begin{itemize}
    \item Git
    \begin{itemize}
        \item Rebasing and Ignoring files
        \item Forking and Special files
        \item Resetting and Tags
        \item Reverting and Automated merges
        \item Hooks and Tags
    \end{itemize}
    \item \LaTeX\ 
    \begin{itemize}
        \item Cross-referencing and Custom commands
        \item Footnotes/margin notes and creating new environments
        \item Floating figures and editing style sheets
        \item Graphics and advanced mathematical equations
        \item Macros and hyperlinks
    \end{itemize}
\end{itemize}

\subsection{Advanced features: add student 1 name here}

Explain your use of the advanced Git and \LaTeX\ features. 

\subsection{Advanced features: add student 2 name here}

Explain your use of the advanced Git and \LaTeX\ features. 

\subsection{Advanced features: add student 3 name here}

Explain your use of the advanced Git and \LaTeX\ features. 

\subsection{Advanced features: add student 4 name here}

Explain your use of the advanced Git and \LaTeX\ features. 



%=======================================================================================

\newpage
\section{Level 4: Advanced Knowledge}

Level 4 focuses on analysing your particular tech stack and considering alternatives. Each student should consider the tech stack they described for Level 2, and then discuss each of the following points:
\begin{itemize}
    \item What are the strengths and limitations of this stack? (Target = $\sim$200 words).
    \item What alternatives exist, and under what situations might these alternatives be a better choice? (Target = $\sim$200 words).
\end{itemize}

\subsection{Advanced Knowledge: add student 1 name here}

Your text goes here

\subsection{Advanced Knowledge: add student 2 name here}

Your text goes here

\subsection{Advanced Knowledge: add student 3 name here}

Your text goes here

\subsection{Advanced Knowledge: add student 4 name here}

Your text goes here



%=======================================================================================

\newpage

\bibliographystyle{apacite}
\bibliography{main}

Ikechukwu, L. (2020). \textit{The Self-Taught Developer's Guide to Learning How to Code.} (See\\ \texttt{https://www.freecodecamp.org/news/the-self-taught-developers-guide-to-coding/})
\\
\\
Colombia Universtiy (2020). \textit{Are Coding Bootcamps Worth It? [What the Numbers Say] - Columbia Engineering Boot Camps} (See\\ \texttt{https://bootcamp.cvn.columbia.edu/blog/are-coding-bootcamps-worth-it/}) 
\\
\\
Shipley, B. (2019). \textit{A Realistic Perspective Of The Pros And Cons Of Coding Bootcamps} (See\\ \texttt{https://medium.com/swlh/a-realistic-perspective-\\
	of-the-pros-and-cons-of-coding-bootcamps-527a1e4b8fb2}) 
\\
\\
Eira, A. (2018). What Is Continuous Learning: Benefits of Adopting It in Your Workplace - Financesonline.com. Retrieved 27 March 2022, from https://financesonline.com/what-is-continuous-learning/
\\
\\
Erstad, W. (2017). Computer Programmer Skills: The Perfect Balance of Hard \& Soft Skills Employers Are Seeking | Rasmussen University. Retrieved 27 March 2022, from https://www.rasmussen.edu/degrees/technology/blog/5-soft-skills-programmers-need/
\\
\\
3 Reasons Why Data Scientists Should Learn Statistics Well. (2022). Retrieved 3 April 2022, from https://towardsdatascience.com/3-reasons-why-data-scientists-should-learn-statistics-well-90e80ae6c68f
\\
\\
9 Must-have skills you need to become a Data Scientist, updated - KDnuggets. (2022). Retrieved 3 April 2022, from https://www.kdnuggets.com/2018/05/simplilearn-9-must-have-skills-data-scientist.html
\\
\\
How Much Do Data Scientists Need to Know about Statistics? — School of Data Science. (2022). Retrieved 3 April 2022, from https://datascience.virginia.edu/news/how-much-do-data-scientists-need-know-about-statistics
\\
\\
Shin, T. (2022). 11 Most Practical Data Science Skills for 2022 - KDnuggets. Retrieved 3 April 2022, from https://www.kdnuggets.com/2021/10/11-most-practical-data-science-skills-2022.html
\\
\\
Top 8 Tech Stacks: Choosing the Right Tech Stack. (2022). Retrieved 3 April 2022, from https://fullscale.io/blog/top-5-tech-stacks/
\\
\\
Unwin, A. (2020). Why is Data Visualization Important? What is Important in Data Visualization?. 2.1. doi: 10.1162/99608f92.8ae4d525
\\
\\
Weihs, C., & Ickstadt, K. (2018). Data Science: the impact of statistics. International Journal Of Data Science And Analytics, 6(3), 189-194. doi: 10.1007/s41060-018-0102-5
\\
\\
Why Data Visualization Is Important - Analytiks. (2022). Retrieved 3 April 2022, from https://analytiks.co/importance-of-data-visualization/


\end{document}
\end{report}
